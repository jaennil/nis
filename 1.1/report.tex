\input{prelude}
\graphicspath{ {1.1/models/} }
\usepackage{hyperref}
\usepackage{upgreek}
\begin{document}
\pagestyle{fancy}
\fancyhead{}
% disable header
\renewcommand{\headrulewidth}{0pt}
\singlespacing

\newpage
\begin{center}
    Министерство науки и высшего образования Российской Федерации
    Федеральное государственное автономное образовательное учреждение

    высшего образования

    \guillemotleft МОСКОВСКИЙ ПОЛИТЕХНИЧЕСКИЙ УНИВЕРСИТЕТ\guillemotright

    (МОСКОВСКИЙ ПОЛИТЕХ)
\end{center}
\noindent
\bigbreak
\bigbreak
\bigbreak
\bigbreak
\begin{center}
    \textbf{СВЕДЕНИЯ ИЗ ТЕОРИИ МАРКОВСКИЙ ПРОЦЕССОВ}
    \bigskip
    \bigskip
    \bigskip
    \bigskip
    \bigskip

    Лабораторная работа 3.1

    По курсу \guillemotleft Надёжность информационных систем\guillemotright
    \bigskip

    \bigbreak
    \bigbreak
    \bigbreak
    \bigbreak
\end{center}
\noindent
\bigbreak
\bigbreak
\bigbreak
\bigbreak
\bigbreak
\bigbreak
\bigbreak
\bigbreak
\bigbreak
\bigbreak
\hfill Выполнил

\hfill Дубровских Н.Е.

\hfill Группа 221-361
\bigbreak
\bigbreak
\bigbreak
\hfill Проверил

\hfill Маковей С.О.
\vfill
\begin{center}
    Москва, 2024
\end{center}
\newpage
\onehalfspacing


\begin{center}
    \textbf{Лабораторная работа 1.1}

    \textbf{Область применения и классификации имитационной моделей. Принципы и этапы имитационного моделирования. Проверка имитационной модели }
\end{center}

К \textbf{основным целям} лабораторной работы следует отнести:

\begin{itemize}
    \item формирование у студентов понимания важности развития и применения средств имитационного моделирования в современных информационных системах и технологиях;
    \item а также ознакомление студентов с основными характеристиками имитационного моделирования.
\end{itemize}

К \textbf{основным задачам} лабораторной работы следует отнести:

\begin{itemize}
    \item развитие навыков анализа состояния и тенденций развития математического и имитационного моделирования;
    \item развитие навыков изучения истории и областей применения методов и систем имитационного моделирования;
    \item развитие навыков классификации средств и характеристик имитационного моделирования.
\end{itemize}
\bigskip

\textbf{Отчет}

Провести расчеты по формулам, учитывая следующие входные условия:

k – каналы обработки запросов, n – емкость буфера, μ -
производительность обработки потока запросов и λ – интенсивность их
входного потока.

\begin{table}[H]
    \small
    \centering
    \setstretch{1}
\begin{tblr}{|c|c|c|c|}
    \hline
    Входные параметры & {Значения при 1\\эксперименте} & {Значения при 2\\эксперименте} & {Значения при 3\\эксперименте}\\
    \hline
    k & 4 & 4 & 4\\
    \hline
n & 3 & 3 & 5\\
\hline
μ & 17 & 10 & 10\\
\hline
λ & 200 & 50 & 200\\
    \hline
\end{tblr}
\end{table}

Для начала необходимо рассчитать коэффициент загрузки:

$$p = \frac{\lambda}{\upmu}$$

Для 1 эксперимента:

$$p = \frac{200}{17} = 11.7647$$

Для 2 эксперимента:

$$p = \frac{50}{10} = 5$$

Для 3 эксперимента:

$$p = \frac{200}{10} = 20$$

Чтобы определить коэффициент загрузки СМО, воспользуемся формулой:

$$p_s = \frac{\rho}{k}$$

Для 1 эксперимента:

$$p_s = \frac{11.7647}{4} = 2.9411$$

Для 2 эксперимента:

$$p_s = \frac{5}{4} = 1.25$$

Для 3 эксперимента:

$$p_s = \frac{20}{4} = 5$$

Для расчета показателя простоя следует воспользоваться следующим
выражением:

$$p_0 = \left[ \sum_{j=0}^{k-1} \frac{p^j}{j!} + \frac{p^k \left(1 - p_s^{N+1}\right)}{k! \left(1 - p_s\right)} \right]^{-1}$$

Для 1 эксперимента:

$$p_0 = 0.00003256221406957215$$

Для 2 эксперимента:

$$p_0 = 0.0053$$

Для 3 эксперимента:

$$p_0 = 0.00000003840016564295$$

После определения простоя следует учесть вероятность отказа в
дальнейшем обслуживании системы

$$p_{den} = p_{(k+n)} = \frac{p^{k+n}p_0}{k!k^n}$$

Для 1 эксперимента:

$$p_{den} = 0.66128506045854$$

Для 2 эксперимента:

$$p_{den} = 0.2695719401041667$$

Для 3 эксперимента:

$$p_{den} = 0.8000034508947916$$

Далее необходимо вычислить среднее количество каналов, которые
остались занятыми и запросов в очереди:

$$k_{ch} = p(1 - p_{den})$$

Для 1 эксперимента:

$$k_{ch} = 3.984879649223415$$

Для 2 эксперимента:

$$k_{ch} = 3.652140299479166$$

Для 3 эксперимента:

$$k_{ch} = 3.9999309821041673$$

Чтобы определить средний показатель запросов, стоящих в очереди,
приводится вычисление по формуле:

$$W_{\text{req}} = \frac{p^{k+1} p_0}{k!k} \left[ \frac{1 - p_s^n (n + 1 - n p_s)}{(1 - p_s)^2} \right]$$

Для 1 эксперимента:

$$W_{req} = 2.509861234688828$$

Для 2 эксперимента:

$$W_{req} = 1.4125569661458333$$

Для 3 эксперимента:

$$W_{req} = 4.750100490032915$$

Далее необходимо узнать среднее время, потраченное системой на
ожидание перед обработкой в очереди Wreq, учитывая интенсивность
полученных запросов:

$$T_{req} = \frac{W_{req}}{\lambda}$$

Для 1 эксперимента:

$$T_{req} = 0.01254930617$$

Для 2 эксперимента:

$$T_{req} = 0.02825113932$$

Для 3 эксперимента:

$$T_{req} = 0.02375050245$$

Теперь можно определить среднее количество запросов в СМО:

$$W_s = k_{ch} + W_{req}$$

Для 1 эксперимента:

$$W_s = 6.49474088391$$

Для 2 эксперимента:

$$W_s = 5.06469726562$$

Для 3 эксперимента:

$$W_s = 8.75003147214$$

Определив $T_{req}$, можно найти среднее время, в течение которого запросы
находятся в СМО:

$$T_s = T_{req} + \frac{(1-p_{den})}{\upmu}$$

Для 1 эксперимента:

$$T_s = 0.03247371437832118$$

Для 2 эксперимента:

$$T_s = 0.10129394530958333$$

Для 3 эксперимента:

$$T_s = 0.04375015736052083$$

\end{document}
