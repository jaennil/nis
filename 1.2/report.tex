\documentclass[14pt]{extarticle}

\usepackage{fontspec}
\setmainfont{Times New Roman}

% размер полей
\usepackage{geometry}
\geometry{a4paper, top=2cm, bottom=2cm, right=1.5cm, left=3cm}

 %debugging
%\usepackage{showframe}

% полуторный интервал
\usepackage{setspace}
\onehalfspacing

% абзацный отступ
\setlength{\parindent}{1.25cm}

% выравнивание текста по ширине
\sloppy

% списки
\usepackage{calc} % арифметические операции с величинами
\usepackage{enumitem}
\setlist{
    nosep,
    leftmargin=0pt,
    itemindent=\parindent + \labelwidth - \labelsep,
}

% подписи к рисункам и таблицам
\usepackage{caption}
\renewcommand{\figurename}{Рисунок}
\renewcommand{\tablename}{Таблица}
\DeclareCaptionFormat{custom}
{
    \textit{#1#2#3}
}
\DeclareCaptionLabelSeparator{custom}{. }
\captionsetup{
    % хз какой это размер - 12 или нет, но выглядит меньше 14
    font=small,
    format=custom,
    labelsep=custom,
}

% картинки
\usepackage{graphicx}

% колонтитулы
\usepackage{fancyhdr}

% картинки и таблицы находятся именно в том месте текста где помещены (атрибут H)
\usepackage{float}

% таблицы
\usepackage{tabularray}

\graphicspath{ {1.2/models/} }
\usepackage{hyperref}
\usepackage{upgreek}
\begin{document}
\pagestyle{fancy}
\fancyhead{}
% disable header
\renewcommand{\headrulewidth}{0pt}
\singlespacing

\newpage
\begin{center}
    Министерство науки и высшего образования Российской Федерации
    Федеральное государственное автономное образовательное учреждение

    высшего образования

    \guillemotleft МОСКОВСКИЙ ПОЛИТЕХНИЧЕСКИЙ УНИВЕРСИТЕТ\guillemotright

    (МОСКОВСКИЙ ПОЛИТЕХ)
\end{center}
\noindent
\bigbreak
\bigbreak
\bigbreak
\bigbreak
\begin{center}
    \textbf{ОСНОВЫ СИСТЕМНОГО АНАЛИЗА И ИМИТАЦИОННОГО МОДЕЛИРОВАНИЯ}
    \bigskip
    \bigskip
    \bigskip
    \bigskip
    \bigskip

    Лабораторная работа 1.2

    По курсу \guillemotleft Надёжность информационных систем\guillemotright
    \bigskip

    \bigbreak
    \bigbreak
    \bigbreak
    \bigbreak
\end{center}
\noindent
\bigbreak
\bigbreak
\bigbreak
\bigbreak
\bigbreak
\bigbreak
\bigbreak
\bigbreak
\bigbreak
\bigbreak
\hfill Выполнил

\hfill Дубровских Н.Е.

\hfill Группа 221-361
\bigbreak
\bigbreak
\bigbreak
\hfill Проверил

\hfill Маковей С.О.
\vfill
\begin{center}
    Москва, 2024
\end{center}
\newpage
\onehalfspacing


\begin{center}
    \textbf{Лабораторная работа 1.2}

    \textbf{Анализ свойств имитационной модели. Анализ
чувствительности модели. Языки имитационного моделирования.
Методика выбора языка моделирования. ММ. Статистическое
моделирование.}
\end{center}

К \textbf{основным целям} лабораторной работы следует отнести:

\begin{itemize}
    \item формирование у студентов понимания свойств и принципов построения имитационной модели;
    \item ознакомление студентов с принципами анализа чувствительности;
    \item ознакомление студентов с назначением и основными свойствами языка имитационного моделирования;
    \item ознакомление студентов с назначением и основными свойствами статистического моделирования.
\end{itemize}

К \textbf{основным задачам} лабораторной работы следует отнести:

\begin{itemize}
    \item анализа состояния и принципов построения имитационной модели;
    \item развитие навыков анализа закономерностей чувствительности имитационной модели;
    \item развитие навыков понимания типовой структуры языка имитационного моделирования;
    \item развитие навыков анализа тенденций развития систем статистического моделирования.
\end{itemize}
\bigskip

\textbf{Отчет}

\begin{table}[H]
    \small
    \centering
    \setstretch{1}
\begin{tblr}{|c|c|c|}
    \hline
    Входные параметры & {Значения при 1\\эксперименте} & {Значения при 2\\эксперименте}\\
    \hline
    k & 5 & 6\\
    \hline
n & 5 & 3\\
\hline
μ & 20 & 10\\
\hline
λ & 200 & 150\\
    \hline
\end{tblr}
\end{table}

Для начала необходимо рассчитать коэффициент загрузки:

$$p = \frac{\lambda}{\upmu}$$

Для 1 эксперимента:

$$p = \frac{200}{20} = 10$$

Для 2 эксперимента:

$$p = \frac{150}{10} = 15$$

Чтобы определить коэффициент загрузки СМО, воспользуемся формулой:

$$p_s = \frac{\rho}{k}$$

Для 1 эксперимента:

$$p_s = \frac{10}{5} = 2$$

Для 2 эксперимента:

$$p_s = \frac{15}{6} = 2.5$$

Для расчета показателя простоя следует воспользоваться следующим
выражением:

$$p_0 = \left[ \sum_{j=0}^{k-1} \frac{p^j}{j!} + \frac{p^k \left(1 - p_s^{N+1}\right)}{k! \left(1 - p_s\right)} \right]^{-1}$$

Для 1 эксперимента:

$$p_0 = 0.0000188167$$

Для 2 эксперимента:

$$p_0 = 0.0000024356$$

После определения простоя следует учесть вероятность отказа в
дальнейшем обслуживании системы

$$p_{den} = p_{(k+n)} = \frac{p^{k+n}p_0}{k!k^n}$$

Для 1 эксперимента:

$$p_{den} = 0.50177866$$

Для 2 эксперимента:

$$p_{den} = 0.602061767578125$$

Далее необходимо вычислить среднее количество каналов, которые
остались занятыми и запросов в очереди:

$$k_{ch} = p(1 - p_{den})$$

Для 1 эксперимента:

$$k_{ch} = 4.9822133$$

Для 2 эксперимента:

$$k_{ch} = 5.969073486328124$$

Чтобы определить средний показатель запросов, стоящих в очереди,
приводится вычисление по формуле:

$$W_{\text{req}} = \frac{p^{k+1} p_0}{k!k} \left[ \frac{1 - p_s^n (n + 1 - n p_s)}{(1 - p_s)^2} \right]$$

Для 1 эксперимента:

$$W_{req} = 4.045590499$$

Для 2 эксперимента:

$$W_{req} = 2.384164599609375$$

Далее необходимо узнать среднее время, потраченное системой на
ожидание перед обработкой в очереди Wreq, учитывая интенсивность
полученных запросов:

$$T_{req} = \frac{W_{req}}{\lambda}$$

Для 1 эксперимента:

$$T_{req} = 0.020227952499$$

Для 2 эксперимента:

$$T_{req} = 0.0158944306640625$$

Теперь можно определить среднее количество запросов в СМО:

$$W_s = k_{ch} + W_{req}$$

Для 1 эксперимента:

$$W_s = 9.027803833$$

Для 2 эксперимента:

$$W_s = 8.3532380859375$$

Определив $T_{req}$, можно найти среднее время, в течение которого запросы
находятся в СМО:

$$T_s = T_{req} + \frac{(1-p_{den})}{\upmu}$$

Для 1 эксперимента:

$$T_s = 0.045139019166$$

Для 2 эксперимента:

$$T_s = 0.05568825390625$$

\pagebreak

\begin{center}
    \textbf{ВЫВОД ПО ПРОВЕДЕННОЙ РАБОТE}
\end{center}

В данном исследовании наиболее существенное влияние на вероятность отказа в обслуживании оказывает интенсивность входного потока запросов (λ), так как она напрямую определяет степень загрузки системы.

В рамках проведенного исследования видно, что
система моделирования работает верно и позволяет на ее основе анализировать и оценивать ключевые показатели работы системы массового обслуживания, такие как вероятность простоя ($p_0$), вероятность отказа ($p_{den}$), среднее количество занятых каналов ($k_{ch}$), среднее количество запросов в очереди ($W_{req}$) и общее время нахождения заявок в системе ($T_s$).

\end{document}
